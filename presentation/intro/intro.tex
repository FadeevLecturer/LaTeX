\section{Введение}

\begin{frame}{\TeX}
	\begin{quotation}
		\TeX~--- система компьютерной вёрстки, разработанная американским профессором информатики Дональдом Кнутом в целях создания компьютерной типографии. В неё входят средства для секционирования документов, для работы с перекрёстными ссылками. В частности, благодаря этим возможностям, TeX популярен в академических кругах, особенно среди математиков и физиков.	
	\end{quotation}
	\begin{flushright}
		\href{https://ru.wikipedia.org/wiki/TeXs}{Wikipedia}
	\end{flushright}
	
\end{frame}

\begin{frame}{\TeX{} vs. MS Word}
	\begin{table}
		\begin{tabular}{p{0.49\textwidth}|p{0.49\textwidth}}
			\TeX & MS Word \\
			\hline
			What you see is what you mean & What you see is what you get \\
			\hline \adv{Созданный документ выглядит одинаково на всех устройствах}& \dis{Созданный документ может выглядеть иначе на другом устройстве} \\
			\hline
			\adv{Удобно набирать формулы} & \dis{Неудобно набирать формулы} \\
			\hline
			\adv{Сравнительно легко организовать автоматическую нумерацию глав, разделов, формул, таблиц, иллюстраций и т.п, а также перекрестные ссылки на них} & \dis{Сравнительно трудоемко организовать автоматическую нумерацию глав, разделов, формул, таблиц, иллюстраций и т.п, а также перекрестные ссылки на них} \\
			\hline \adv{Содержимое документа и его форматирование изолированы друг от друга} & \dis{Содержимое документа и его форматирование сложно изолировать друг от друга}
		\end{tabular}
	\end{table}
\end{frame}

\begin{frame}{\TeX{} vs. MS Word}
	\begin{table}
		\begin{tabular}{p{0.49\textwidth}|p{0.49\textwidth}}
			\TeX & MS Word \\
			\hline
			What you see is what you mean & What you see is what you get \\
			\hline \dis{Просмотр текста и его набор~--- разные операции} & \adv{Текст одновременно набирается и просматривается} \\
			\hline \dis{Трудно набирать страницы со сложным макетом} & \adv{Макет страницы можно настраивать налету}
		\end{tabular}
	\end{table}
\end{frame}


\begin{frame}[fragile]{\TeX{} из командной строки}
	Для работы с \TeX{} необходимо установить дистрибутив, например, \href{https://miktex.org/}{MiKTeX} или \href{https://www.tug.org/texlive/}{TeXLive}. 
	
	После установки дистрибутива исходный текст можно скомпилировать в ``\textbf{.dvi}'' командой
	
	\large
	\begin{minted}[autogobble]{ps1}
		$ latex helloworld.tex
	\end{minted}
	
	\[
		\textbf{helloworld.tex} \longrightarrow \text{\LaTeX} \longrightarrow \textbf{helloworld.dvi}
	\]
	
	А в ``\textbf{.pdf}'' командой
	
	\begin{minted}[autogobble]{ps1}
		$ pdflatex helloworld.tex
	\end{minted}
	
	\[
	\textbf{helloworld.tex} \longrightarrow \text{pdf\LaTeX} \longrightarrow \textbf{helloworld.pdf}
	\]
\end{frame}

\begin{frame}[fragile]{Редактор \LaTeX{}}
	При этом набирать исходный текст можно почти в любом текстовом редакторе. Однако гораздо удобнее установить специальный редактор под \LaTeX. 
	
	Автор предпочитает редактор \href{https://www.texstudio.org/}{TeXstudio}, который поддерживает проверку и подсветку синтаксиса, автоматическое дополнение команд, выделение ошибок, предварительный просмотр и многое другое. Если установить дистрибутив \LaTeX до установки \textit{TeXstudio}, то редактор сам найдет дистрибутив и настроит трансляцию исходного кода документа в \textbf{pdf} файл.
	
	\begin{alertblock}{Предупреждение}
		По умолчанию \textit{TeXstudio} ничего не знает про русский язык и будет подсвечивать каждое слово написанное на кириллице, как грамматическую ошибку. Чтобы это исправить, необходимо установить русский словарь. Инструкция доступна по ссылке. 
	\end{alertblock}
\end{frame}


\begin{frame}[fragile]{Простейший документ}
	Простейший документ выглядит примерно так.
	\begin{minted}[autogobble, tabsize=2, frame=single]{LaTeX}
		\documentclass{article}
		\begin{document}
			Hello world, \LaTeX.
		\end{document}
	\end{minted}

\end{frame}

\begin{frame}[fragile]{Документ и преамбула}
	\begin{itemize}
		\item Все, что идёт после \mintinline{TeX}{\documentclass{article}} но до \mintinline{TeX}{\begin{document}}, называется преамбулой документа.
		\item Все, что идет после \mintinline{TeX}{\begin{document}} но до \mintinline{TeX}{\end{document}}, задает содержимое документа.
		\item Все, что идет после символа \%, считается комментарием и игнорируется \LaTeX{} до конца этой строки.
	\end{itemize}
	
	
	
	
	\begin{minted}[autogobble, tabsize=2, frame=single]{LaTeX}
		\documentclass{article}
		% preambule
		% комментарий в преамбуле
		\begin{document}
			% content
			% комментарий в документе
		\end{document}
	\end{minted}
	
\end{frame}


\begin{frame}[fragile]{Преамбула}
	В преамбуле задаётся стиль документа и подключаются пакеты. Например, для того, чтобы использовать кириллицу в \LaTeX, необходимо подключить пару пакетов.
	
	\begin{minted}[autogobble, tabsize=2, frame=single]{LaTeX}
		\documentclass{article}
		
		\usepackage[T2A]{fontenc}
		\usepackage[utf8]{inputenc}
		\usepackage[russian]{babel}
		
		\begin{document}
			Привет мир, \LaTeX.
		\end{document}
	\end{minted}
		
\end{frame}


\begin{frame}[fragile]{Преамбула}
	Ещё пара пакетов имеет смысл импортировать по-умолчанию, если вы планируете набирать формулы. 
	
	\begin{minted}[autogobble, tabsize=2, frame=single, fontsize=\small]{LaTeX}
		\documentclass{article}
		
		\usepackage[T2A]{fontenc}
		\usepackage[utf8]{inputenc}
		\usepackage[russian]{babel}
		
		\usepackage{amsmath}
		\usepackage{amssymb}
		
		\begin{document}
			Привет мир, \LaTeX.
		\end{document}
	\end{minted}
	
\end{frame}


\begin{frame}[fragile]{Набор текста}
	\begin{minipage}{0.49\textwidth}
		\begin{minted}[autogobble, tabsize=2, frame=single]{LaTeX}
			Слова   разделяются пробелами,
			а           абзацы ---
			        пустыми строками.
			
			Абзацный отступ в исходном
			      тексте     оставлять
			             не
			        надо: он получается
			автоматически.
		\end{minted}
	\end{minipage}
	\begin{minipage}{0.49\textwidth}
		\parindent=1cm
		Слова разделяются пробелами,
		а абзацы ---
		пустыми строками.
		
		Абзацный отступ в исходном
		тексте оставлять
		не
		надо: он получается
		автоматически.
	\end{minipage}
	
\end{frame}


\begin{frame}[fragile]{Специальные символы}
	\large
	\begin{table}
	\begin{tabular}{|c|c|}
		% first row
		\hline \Verb|{| & Начало группы \\
		\hline \Verb|}| & Конец группы \\
		\hline \Verb|$| & Начало или конец формулы \\
		\hline \Verb|#| & Аргумент при определении пользовательской команды \\
		\hline \Verb|&| & Отделение элементов матрицы или таблицы \\
		\hline \Verb|%| & Комментарий \\
		\hline \Verb|_| & Нижний индекс в формуле \\
		\hline \Verb|^| & Верхний индекс в формуле \\
		\hline \Verb|~| & Неразрывный пробел \\
		\hline \Verb|\| & Начало команды \\
		\hline
	\end{tabular}
	\caption{Специальные символы}
\end{table}
	
\end{frame}

\begin{frame}[fragile]{Простейшие команды}
	Команды начинаются с символа ``\mintinline{TeX}{\}''. Самые простейшие команды позволяют напечатать некоторые специальные символы в их исходном виде. 
	
	\begin{minipage}{0.49\textwidth}
		\begin{minted}[autogobble, tabsize=2, frame=single]{LaTeX}
			Курс тугрика повысился на 7\%, 
			и теперь за него дают \$200.
		\end{minted}
	\end{minipage}
	\begin{minipage}{0.49\textwidth}
		Курс тугрика повысился на 7\%, и теперь за него дают \$200.
	\end{minipage}

	Остальные команды состоят из символов латинского алфавита. Например, команда \mintinline{TeX}{\TeX} печатает символ \TeX. Ещё, например, есть команды меняющие шрифт текста.
	
	
	\begin{minted}[autogobble, tabsize=2, frame=single]{LaTeX}
		Этот текст обычного размера, \Large а этот более крупного.
	\end{minted}
	
	Этот текст обычного размера, \Large а этот более крупного.
\end{frame}

\begin{frame}[fragile]{Простейшие команды}
	Команды начинаются с символа ``\mintinline{TeX}{\}''. Самые простейшие команды позволяют напечатать некоторые специальные символы в их исходном виде. 
		
	\begin{minipage}{0.49\textwidth}
		\begin{minted}[autogobble, tabsize=2, frame=single]{LaTeX}
			Курс тугрика повысился на 7\%, 
			и теперь за него дают \$200.
		\end{minted}
	\end{minipage}
	\begin{minipage}{0.49\textwidth}
		Курс тугрика повысился на 7\%, и теперь за него дают \$200.
	\end{minipage}
	
	Остальные команды состоят из символов латинского алфавита. Например, команда \mintinline{TeX}{\TeX} печатает символ \TeX. Ещё, например, есть команды меняющие шрифт текста.
	
	\begin{minted}[autogobble, tabsize=2, frame=single, fontsize=\small]{LaTeX}
		Полужирный шрифт начнется с \bfseries этого слова. Снова \mdseries светлый, 
		теперь \slshape наклонный, до нового переключения; вновь \upshape прямой..
	\end{minted}
	
	Полужирный шрифт начнется с \bfseries этого слова. Снова \mdseries светлый, теперь \slshape наклонный, до нового переключения; вновь \upshape прямой..
\end{frame}


\begin{frame}[fragile]{Простейшие команды}
	Команды начинаются с символа ``\mintinline{TeX}{\}''. Самые простейшие команды позволяют напечатать некоторые специальные символы в их исходном виде. 
		
	\begin{minipage}{0.49\textwidth}
		\begin{minted}[autogobble, tabsize=2, frame=single]{LaTeX}
			Курс тугрика повысился на 7\%, 
			и теперь за него дают \$200.
		\end{minted}
	\end{minipage}
	\begin{minipage}{0.49\textwidth}
		Курс тугрика повысился на 7\%, и теперь за него дают \$200.
	\end{minipage}
	
	Остальные команды состоят из символов латинского алфавита. Например, команда \mintinline{TeX}{\TeX} печатает символ \TeX. Ещё, например, есть команды меняющие шрифт текста.
	
	\begin{minted}[autogobble, tabsize=2, frame=single, fontsize=\small]{LaTeX}
		Полужирный шрифт начнется с \bfseries этого слова. Снова \mdseries светлый, 
		теперь \slshape наклонный, до нового переключения; вновь \upshape прямой..
	\end{minted}
	
	Полужирный шрифт начнется с \bfseries этого слова. Снова \mdseries светлый, теперь \slshape наклонный, до нового переключения; вновь \upshape прямой.
\end{frame}


\begin{frame}[fragile]{Группы}
	Группы обозначаются парой фигурных скобок. Некоторые команды действуют в группах локально.
	
	\begin{minipage}{0.49\textwidth}
		\begin{minted}[autogobble, tabsize=2, frame=single]{LaTeX}
			Полужирным шрифтом набрано только 
			{\bfseries это} слово; после 
			скобок все идет, как прежде.
		\end{minted}
	\end{minipage}
	\begin{minipage}{0.49\textwidth}
		Полужирным шрифтом набрано только {\bfseries это}
		слово; после скобок все идет, как прежде.
	\end{minipage}

	Как правило, по выходу из группы модифицирующие команды сбрасываются.
	
	\begin{minipage}{0.54\textwidth}
		\begin{minted}[autogobble, tabsize=2, frame=single]{LaTeX}
			Сначала {переключим шрифт на 
			\itshape курсив; теперь сделаем 
			шрифт еще и {\bfseries полужирным;}
			посмотрите, как восстановится} шрифт
			после кон{ца г}руппы.
		\end{minted}
	\end{minipage}
	\begin{minipage}{0.45\textwidth}
		Сначала {переключим шрифт на 
		\itshape курсив; теперь сделаем 
		шрифт еще и {\bfseries полужирным;}
		посмотрите, как восстановится} шрифт
	после кон{ца г}руппы.
	\end{minipage}
	
\end{frame}


\begin{frame}[fragile]{Команды с аргументами}
	Обязательные аргументы указываются у команды в фигурных скобках, необязательные в квадратных.
	
	\begin{minted}[autogobble, tabsize=2, frame=single]{LaTeX}
		\documentclass[12pt,twocolumn]{book}
	\end{minted}
	
	Здесь 
	\begin{itemize}
		\item \mintinline{TeX}{\documentclass}~--- команда;
		\item \mintinline{TeX}{book}~--- обязательный аргумент;
		\item \textcolor{optargs}{\mintinline{TeX}{12pt}} и \textcolor{optargs}{\mintinline{TeX}{twocolumn}}~--- необязательные аргументы;
	\end{itemize}
	
\end{frame}

\begin{frame}[fragile]{Окружения}
	Ещё одна важная конструкция в \LaTeX~--- окружения (environment). 
	
	Окружение начинается с команды \mintinline{TeX}{\begin{имя_окружения}}, а заканчивается командой \mintinline{TeX}{\end{имя_окружения}}. Например, все содержимое документа задаётся в окружении \mintinline{TeX}{document}.
	
	\begin{minipage}{0.69\textwidth}
		\begin{minted}[autogobble, tabsize=2, frame=single]{LaTeX}
			\begin{center}
				Все строки этого абзаца будут центрированы; 
				переносов не будет, если только какое-то 
				слово, как в дезоксирибонуклеиновой кислоте, 
				не длинней строки.
			\end{center}
		\end{minted}
	\end{minipage}
	\begin{minipage}{0.29\textwidth}
		\begin{center}
				Все строки этого абзаца будут
				центрированы; переносов не будет,
				если только какое-то слово,
				как в дезоксирибонуклеиновой
				кислоте, не длинней строки.
		\end{center}
	\end{minipage}

\end{frame}



\begin{frame}[fragile]{Автоматическая генерация ссылок}
	Команда \mintinline{TeX}{\label} создаёт метку, на которую потом можно создать перекрестную ссылку. Команда \mintinline{TeX}{\pageref} подставит номер страницы, на которой располагается указанная метка.
	
	\begin{minted}[autogobble, tabsize=2, frame=single]{LaTeX}
		\textbf{Теорема Пифагора.}\label{pythagoras} В прямоугольном 
		треугольнике квадрат гипотенузы равен сумме квадратов катетов.
		
		Из теоремы Пифагора (см. с.~\pageref{pythagoras}) получаем, что длина
		гипотенузы равна $\sqrt{3^2 + 4^2}=5$.	
	\end{minted}
	
	\textbf{Теорема Пифагора.}\label{pythagoras} В прямоугольном треугольнике квадрат гипотенузы равен сумме квадратов катетов.
	
	Из теоремы Пифагора (см. с.~\ref{pythagoras}) получаем, что длина гипотенузы равна $\sqrt{3^2 + 4^2}=5$.
	
\end{frame}