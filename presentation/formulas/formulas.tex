\section{Набор формул}

\subsection{Вид формул}

\begin{frame}[fragile]{Включенные и выключенные формулы}
	
	Формулы бывают внутри текста (<<включенные>>) и выделенные в отдельную строку (<<выключенные>>).
	
	$\bullet$ \textbf{Включенные} формулы набираются между одинарных знаков \mintinline{TeX}{$} или между пары скобок вида \mintinline{TeX}{\(\)}.
	
	\begin{minted}[autogobble, frame=single]{TeX}
		Оказалось, что $y$ вдвое больше, чем \(x\): $y=2x$.
	\end{minted}
	Оказалось, что $y$ вдвое больше, чем \(x\): $y=2x$.
	
	$\bullet$ \textbf{Выключенные} формулы набираются, например, между двойных знаков \mintinline{TeX}{$$}.
	
	\begin{minted}[autogobble, frame=single]{TeX}
		Оказалось, что $y$ вдвое больше, чем $x$: $$y=2x.$$
	\end{minted}
	Оказалось, что $y$ вдвое больше, чем $x$: $$y=2x.$$
	
\end{frame}

\begin{frame}[fragile]{Выключенные формулы}
	
	Также выключенные формулы можно набирать с помощью скобок \mintinline{TeX}{\[\]}. 
	
	\begin{minipage}{0.49\textwidth}
		\begin{minted}[autogobble, tabsize=4, frame=single]{TeX}
			Оказалось, что $y$ вдвое больше, 
			чем $x$: \[y=2x.\]
		\end{minted}
	\end{minipage}
	\begin{minipage}{0.49\textwidth}
		Оказалось, что $y$ вдвое больше, чем $x$: \[y=2x.\]
	\end{minipage}

	
	\noindent\rule{\textwidth}{1pt}
	А также окружений \mintinline{TeX}{equation} и \mintinline{TeX}{equation*}.
	
	\begin{minipage}{0.49\textwidth}
		\begin{minted}[autogobble, tabsize=4, frame=single]{TeX}
			Оказалось, что $y$ вдвое больше, 
			чем $x$: 
			\begin{equation*}
				y=2x.
			\end{equation*}
		\end{minted}
	\end{minipage}
	\begin{minipage}{0.49\textwidth}
		Оказалось, что $y$ вдвое больше, 
		чем $x$: 
		\begin{equation*}
			y=2x.
		\end{equation*}
	\end{minipage}
	
	
\end{frame}


\begin{frame}[fragile]{Автоматическая нумерация выключенных формул}

	Окружение \mintinline{Tex}{equation} без звездочки автоматически нумерует формулу, что позволяет делать перекрестные ссылки, если поставить внутри формулы метку.
	
	\begin{minted}[autogobble, tabsize=4, frame=single, fontsize=\small]{TeX}
		Оказалось, что $y$ вдвое больше, чем $x$: 
		\begin{equation}\label{eq}
			y=2x.
		\end{equation}
	
		Если $x=0.5$, то по формуле \eqref{eq} со страницы \pageref{eq} 
		находим, что $y=1$.
	\end{minted}
	\label{eqpage}Оказалось, что $y$ вдвое больше, чем $x$: 
	\begin{equation}\label{eq}
		y=2x.
	\end{equation}
	
	Если $x=0.5$, то по формуле \eqref{eq} со страницы \ref{eqpage} находим, что $y=1$.
\end{frame}

%\subsection{Индексы и степени}

\begin{frame}[fragile]{Индексы}
	\Large
	
	Нижние индексы в формулах указываются после символа ``\mintinline{TeX}{_}''.
	
	\begin{minipage}{0.49\textwidth}
		\mintinline{LaTeX}{$x_1$}
	\end{minipage}
	\begin{minipage}{0.49\textwidth}
		\begin{center}
			$x_1$
		\end{center}
	\end{minipage}
	\noindent\rule{\textwidth}{1pt}
	\begin{minipage}{0.49\textwidth}
		\mintinline{LaTeX}{$x_i+1$}
	\end{minipage}
	\begin{minipage}{0.49\textwidth}
		\begin{center}
			$x_i+1$
		\end{center}
	\end{minipage}
	\noindent\rule{\textwidth}{1pt}
	\begin{minipage}{0.49\textwidth}
		\mintinline{LaTeX}{$x_{i+1}$}
	\end{minipage}
	\begin{minipage}{0.49\textwidth}
		\begin{center}
			$x_{i+1}$
		\end{center}
	\end{minipage}
	\noindent\rule{\textwidth}{1pt}
	\begin{minipage}{0.49\textwidth}
		\mintinline{LaTeX}{$x_{i_j}$}
	\end{minipage}
	\begin{minipage}{0.49\textwidth}
		\begin{center}
			$x_{i_j}$
		\end{center}
	\end{minipage}
\end{frame}

\begin{frame}[fragile]{Степени}
	\Large
	
	Степени или верхние индексы в формулах указываются после символа ``\mintinline{TeX}{^}''.
	
	\begin{minipage}{0.49\textwidth}
		\mintinline{LaTeX}{$x^2$}
	\end{minipage}
	\begin{minipage}{0.49\textwidth}
		\begin{center}
			$x^2$
		\end{center}
	\end{minipage}
	\noindent\rule{\textwidth}{1pt}
	\begin{minipage}{0.49\textwidth}
		\mintinline{LaTeX}{$x^2n$}
	\end{minipage}
	\begin{minipage}{0.49\textwidth}
		\begin{center}
			$x^2n$
		\end{center}
	\end{minipage}
	\noindent\rule{\textwidth}{1pt}
	\begin{minipage}{0.49\textwidth}
		\mintinline{LaTeX}{$x^{2n}$}
	\end{minipage}
	\begin{minipage}{0.49\textwidth}
		\begin{center}
			$x^{2n}$
		\end{center}
	\end{minipage}
	\noindent\rule{\textwidth}{1pt}
	\begin{minipage}{0.49\textwidth}
		\mintinline{LaTeX}{$x^{n^2}$}
	\end{minipage}
	\begin{minipage}{0.49\textwidth}
		\begin{center}
			$x^{n^2}$
		\end{center}
	\end{minipage}
\end{frame}


\begin{frame}[fragile]{Степени и индексы. Примеры}
	
	\begin{minted}[autogobble, frame=single]{TeX}
		Катеты $a$, $b$ треугольника связаны с его гипотенузой $c$
		формулой $c^2=a^2+b^2$ (см. с.~\pageref{pythagoras}).
	\end{minted}
	
	Катеты $a$, $b$ треугольника связаны с его гипотенузой $c$
	формулой $c^2=a^2+b^2$ (см. с.~\ref{pythagoras}).
	
	\begin{minted}[autogobble, frame=single]{TeX}
		Из теоремы Ферма следует, что уравнение \[x^{4357}+y^{4357}=z^{4357}\]
		не имеет решений в натуральных числах.
	\end{minted}
	
	Из теоремы Ферма следует, что уравнение
	\[
	x^{4357}+y^{4357}=z^{4357}
	\]
	не имеет решений в натуральных числах.
\end{frame}


\begin{frame}[fragile]{Степени и индексы. Примеры}
	
	\begin{minted}[autogobble, frame=single]{TeX}
		Катеты $a$, $b$ треугольника связаны с его гипотенузой $c$
		формулой $c^2=a^2+b^2$ (см. с.~\pageref{pythagoras}).
	\end{minted}
	
	Катеты $a$, $b$ треугольника связаны с его гипотенузой $c$
	формулой $c^2=a^2+b^2$ (см. с.~\ref{pythagoras}).
	
	\begin{minted}[autogobble, frame=single]{TeX}
		Из теоремы Ферма следует, что уравнение \[x^{4357}+y^{4357}=z^{4357}\]
		не имеет решений в натуральных числах.
	\end{minted}
	
	Из теоремы Ферма следует, что уравнение
	\[
	x^{4357}+y^{4357}=z^{4357}
	\]
	не имеет решений в натуральных числах.
\end{frame}

\begin{frame}[fragile]{Дроби}
	
	Самая простая дробь записывается с помощью знака деления ``\mintinline{TeX}{/}''.
	
	\begin{minipage}{0.39\textwidth}
		\begin{minted}[autogobble, frame=single]{TeX}
			Решая уравнение $2x=1$, 
			получаем $x=1/2$. 
		\end{minted}
	\end{minipage}
	\begin{minipage}{0.59\textwidth}
		Решая уравнение $2x=1$, получаем $x=1/2$. 
	\end{minipage}
	
	Для записи дробей используются команды \mintinline{TeX}{\frac} имеющая два обязательных параметра: числитель и знаменатель дроби. \mintinline{TeX}{\frac} сжимает дробь, чтобы она не вылезала за пределы строки, её аналог \mintinline{TeX}{\dfrac} нет, но это может привести к увеличению междустрочного интервала.
	
	\begin{minipage}{0.39\textwidth}
		\begin{minted}[autogobble, frame=single]{TeX}
			Решая уравнение $2x=1$, 
			получаем $x=\frac{1}{2}$. 
		\end{minted}
	\end{minipage}
	\begin{minipage}{0.59\textwidth}
		Решая уравнение $2x=1$, получаем $x=\frac{1}{2}$. 
	\end{minipage}

	\begin{minipage}{0.39\textwidth}
		\begin{minted}[autogobble, frame=single]{TeX}
			Решая уравнение $2x=1$, 
			получаем $x=\dfrac{1}{2}$. 
		\end{minted}
	\end{minipage}
	\begin{minipage}{0.59\textwidth}
		Решая уравнение $2x=1$, получаем $x=\dfrac{1}{2}$. 
	\end{minipage}
\end{frame}


\begin{frame}[fragile]{Корни}
	\Large
	Корни набираются с помощью команды \mintinline{TeX}{\sqrt}. Обязательный аргумент~--- подкоренное выражение, необязательный аргумент~--- показатель корня.
	
	\begin{minted}[autogobble, frame=single]{TeX}
		По общепринятому соглашению, $\sqrt[3]{x^3}=x$, 
		но $\sqrt{x^2}=|x|$.
	\end{minted}

	По общепринятому соглашению, $\sqrt[3]{x^3}=x$, но $\sqrt{x^2}=|x|$.

\end{frame}

\begin{frame}[fragile]{Штрихи}
	Штрихи (например, для обозначения производной) вводятся символом одинарной кавычки ``\textcolor{Green}{\mintinline{TeX}{'}}''.
	
	\begin{minipage}{0.59\textwidth}
	\begin{minted}[autogobble, frame=single]{TeX}
		Согласно формуле Лейбница,
		\[
		(fg)''=f''g+2f'g'+fg''.
		\]
		Это похоже на формулу квадрата суммы.
	\end{minted}
	\end{minipage}
	\begin{minipage}{0.39\textwidth}
		Согласно формуле Лейбница,
		\[
		(fg)''=f''g+2f'g'+fg''.
		\]
		Это похоже на формулу квадрата суммы.
	\end{minipage}

	Команда \mintinline{TeX}{\prime} печатает такой  штрих: $\prime$. Постановка одинарной кавычки в формуле эквивалента возведению в степень \mintinline{TeX}{\prime}. 
	
	\begin{minipage}{0.49\textwidth}
		\mintinline{LaTeX}{$x'$}
	\end{minipage}
	\begin{minipage}{0.49\textwidth}
		\begin{center}
			$x'$
		\end{center}
	\end{minipage}
	\noindent\rule{\textwidth}{1pt}
	\begin{minipage}{0.49\textwidth}
		\mintinline{LaTeX}{$x^\prime$}
	\end{minipage}
	\begin{minipage}{0.49\textwidth}
		\begin{center}
			$x^\prime$
		\end{center}
	\end{minipage}
\end{frame}

\begin{frame}[fragile]{Многоточия}
	Команда \mintinline{TeX}{\cdots} печатает символы по центру строки ($\cdots$), а команда \mintinline{TeX}{\ldots} снизу ($\ldots$).
	
	\begin{minipage}{0.59\textwidth}
		\begin{minted}[autogobble, frame=single]{TeX}
			В~детстве К.-Ф.~Гаусс придумал,
			как быстро найти сумму
			\[
			1+2+\cdots+100=5050;
			\]
			это случилось, когда школьный
			учитель задал классу найти
			сумму чисел $1,2,\ldots,100$.
		\end{minted}
	\end{minipage}
	\begin{minipage}{0.39\textwidth}
		В~детстве К.-Ф.~Гаусс придумал,
		как быстро найти сумму
		$$
		1+2+\cdots+100=5050;
		$$
		это случилось, когда школьный
		учитель задал классу найти
		сумму чисел $1,2,\ldots,100$.
	\end{minipage}
	
\end{frame}


\begin{frame}[fragile]{Общее для формул}
	
	\begin{itemize}
		\item Все буквы в формуле воспринимаются как отдельные символы и по умолчанию пишутся наклонным курсивным шрифтом:  
		
		\begin{minipage}{0.44\textwidth}
			\mintinline{LaTeX}{Сравните sin и $sin$.}
		\end{minipage}
		\begin{minipage}{0.44\textwidth}
			\begin{center}
				Сравните sin и $sin$.
			\end{center}
		\end{minipage} 
		
		\item Менять шрифт в формуле можно, например, командой \mintinline{TeX}{\mathrm}:
		
		\begin{minipage}{0.44\textwidth}
			\mintinline{LaTeX}{Сравните $sin$ и $\mathrm{sin}$.}
		\end{minipage}
		\begin{minipage}{0.44\textwidth}
			\begin{center}
				Сравните $sin$ и $\mathrm{sin}$.
			\end{center}
		\end{minipage}
		
		\item \LaTeX{} игнорирует все пробелы в формулах.
		
		\begin{minipage}{0.44\textwidth}
			\begin{minted}[autogobble, frame=single]{TeX}
				Формула $y      =   2      x$
				~--- тоже самое, что и 
				формула $y=2x$. 
			\end{minted}
		\end{minipage}
		\begin{minipage}{0.44\textwidth}
				Формула $y      =   2      x$
				~--- тоже самое, что и 
				формула $y=2x$.
		\end{minipage}
		\item Можно расставлять пробелы разного размера в формулах в ручную командами 
		\mintinline{TeX}{\quad, \qquad, \,,\:, \;, \!}.
		
		\item В выключенным формулах \LaTeX{} игнорирует все переносы строк. В включенных формулах допускается один перенос подряд.
	\end{itemize}

	 
	
	
\end{frame}


\subsection{Символы греческого алфавита}

\begin{frame}{Символы греческого алфавита}
	
	Для каждого символа греческого алфавита есть своя команда.
	
	\large
	
	\begin{table}
	\begin{tabular}{|c|c||c|c||c|c||c|c||c|c|}
		% first row
		\hline $\alpha$ & \Verb|\alpha| 
		& $\beta$ & \Verb|\beta| 
		& $\gamma$ & \Verb|\gamma| 
		& $\delta$ & \Verb|\delta| 
		& $\epsilon$ & \Verb|\epsilon| \\
		% second row
		\hline $\zeta$ & \Verb|\zeta| 
		& $\eta$ & \Verb|\eta| 
		& $\theta$ & \Verb|\theta| 
		& $\iota$ & \Verb|\iota| 
		& $\kappa$ & \Verb|\kappa| \\
		% third row
		\hline $\lambda$ & \Verb|\lambda| 
		& $\mu$ & \Verb|\mu| 
		& $\nu$ & \Verb|\nu| 
		& $\xi$ & \Verb|\xi| 
		& $\pi$ & \Verb|\pi| \\
		% fourth row
		\hline $\rho$ & \Verb|\rho| 
		& $\sigma$ & \Verb|\sigma| 
		& $\tau$ & \Verb|\tau| 
		& $\upsilon$ & \Verb|\upsilon|
		& $\phi$ & \Verb|\phi| \\
		% fifth row
		\hline $\chi$ & \Verb|\chi| 
		& $\psi$ & \Verb|\psi| 
		&&&&&&\\
		\hline
	\end{tabular}
	\caption{Строчные символы греческого алфавита}
\end{table}
		
	\normalfont
	
	\begin{exampleblock}{Замечание}
		Эти команды работают только внутри формул.
	\end{exampleblock}
	
\end{frame}


\begin{frame}[fragile=singleslide]{Символы греческого алфавита}

	\begin{minted}[autogobble, frame=single]{TeX}
		Пусть $\alpha$, $\beta$ и $\gamma$~--- углы треугольника, тогда
		\[
		\alpha + \beta + \gamma = \pi.
		\]
	\end{minted}

	\Large
	Пусть $\alpha$, $\beta$ и $\gamma$~--- углы треугольника, тогда
	\[
		\alpha + \beta + \gamma = \pi.
	\]
	
\end{frame}	

\begin{frame}{Символы греческого алфавита}
	Некоторые символы имеют несколько вариантов написания.
	
	\large
	
	\begin{table}
	\begin{tabular}{|c|c||c|c|}
		\hline $\phi$ & \Verb|\phi| & $\varphi$ & \Verb|\varphi| \\
		\hline $\epsilon$ & \Verb|\epsilon| & $\varepsilon$ & \Verb|\varepsilon| \\
		\hline $\kappa$ & \Verb|\kappa| & $\varkappa$ & \Verb|\varkappa| \\
		\hline $\rho$ & \Verb|\rho| & $\varrho$ & \Verb|\varrho| \\
		\hline $\pi$ & \Verb|\pi| & $\varpi$ & \Verb|\varpi| \\
		\hline $\sigma$ & \Verb|\sigma| & $\varsigma$ & \Verb|\varsigma| \\
		\hline $\theta$ & \Verb|\theta| & $\vartheta$ & \Verb|\vartheta| \\
		\hline
	\end{tabular}
	\caption{Строчные символы греческого алфавита}
\end{table}
	
\end{frame}

\begin{frame}{Символы греческого алфавита}
	Команды для прописных символов греческого алфавита начинаются с прописных букв, если их начертание не совпадает с прописными символами латинского алфавита.
	\large
	
	\begin{table}
	\begin{tabular}{|c|c||c|c||c|c||c|c||c|c|}
		% first row
		\hline $\Gamma$ & \Verb|\Gamma| 
		& $\Delta$ & \Verb|\Delta| 
		& $\Theta$ & \Verb|\Theta| 
		& $\Lambda$ & \Verb|\Lambda| 
		& $\Xi$ & \Verb|\Xi| \\
		% second row
		\hline $\Pi$ & \Verb|\Pi| 
		& $\Sigma$ & \Verb|\Sigma| 
		& $\Upsilon$ & \Verb|\Upsilon| 
		& $\Phi$ & \Verb|\Phi| 
		& $\Psi$ & \Verb|\Psi| \\
		% third row
		\hline $\Omega$ & \Verb|\Omega| 
		&&&&&&&&\\
		\hline
	\end{tabular}
	\caption{Прописные символы греческого алфавита}
\end{table}
	
\end{frame}


\begin{frame}[fragile]{Символы греческого алфавита}

	\begin{minted}[autogobble, frame=single]{TeX}
		Начертание прописного символа $\alpha$ совпадает с заглавной буквой
		латинского алфавита $A$. Символа $\sigma$ в прописном варианте имеет
		вид $\Sigma$ и не имеет аналога в латинском алфавите. 
		
		В \TeX{} символы $\Sigma$ и $\Pi$ не используются для обозначения
		операторов суммирования и произведения. 
	\end{minted}

	Начертание прописного символа $\alpha$ совпадает с заглавной буквой
	латинского алфавита $A$. Символа $\sigma$ в прописном варианте имеет
	вид $\Sigma$ и не имеет аналога в латинском алфавите. 
	
	В \TeX{} символы $\Sigma$ и $\Pi$ не используются для обозначения
	операторов суммирования и произведения.
\end{frame}

\subsection{Бинарные операторы, операторы сравнения и стрелки}

\begin{frame}[fragile=singleslide]{Бинарные операторы, операторы сравнения и стрелки}
	\begin{center}
		\begin{table}
	\begin{tabular}{|c|c||c|c||c|c||c|c|}
		% first row
		\hline $+$ & \Verb|+| 
		& $-$ & \Verb|-| 
		& $\pm$ & \Verb|\pm| 
		& $\mp$ & \Verb|\mp| \\
		% second row
		\hline $*$ & \Verb|*|
		& $\times$ & \Verb|\times| 
		& $\cdot$ & \Verb|\cdot| 
		& $\circ$ & \Verb|\circ| \\
		% third row
		\hline $\div$ & \Verb|\div| 
		& $\colon$ & \Verb|\colon| 
		& $\setminus$ & \Verb|\setminus|  
		& $/$ & \Verb|/| \\
		% fourth row
		\hline $\cup$ & \Verb|\cup|
		& $\cap$ & \Verb|\cap| 
		& $\vee$ & \Verb|\vee| 
		& $\wedge$ & \Verb|\wedge| \\ 
		% fifth row
		\hline $=$ & \Verb|=| 
		& $\ne$ & \Verb|\ne| 
		& $\equiv$ & \Verb|\equiv|
		& $\doteq$ & \Verb|\doteq| \\
		% sixth row
		\hline $\sim$ & \Verb|\sim| 
		& $\approx$ & \Verb|\approx| 
		& $\propto$ & \Verb|\propto| 
		& $\cong$ & \Verb|\cong| \\
		% & $\in$ & \Verb|\in| \\ 
		% seventh row
		\hline $\gg$ & \Verb|\gg| 
		& $\ll$ & \Verb|\ll|
		& $<$ & \Verb|<| 
		& $>$ & \Verb|>| \\ 
		% eigth row
		\hline $\le$ & \Verb|\le| 
		& $\ge$ & \Verb|\ge| 
		& $\geqslant$ & \Verb|\geqslant| 
		& $\leqslant$ & \Verb|\leqslant| \\
		% nineth row
		\hline $\perp$ & \Verb|\perp|
		& $\in$ & \Verb|\in| 
		& $\notin$ & \Verb|\notin| 
		& $\ni$ & \Verb|\ni| \\  
		% tenth row
		\hline $\subset$ & \Verb|\subset|
		& $\subseteq$ & \Verb|\subseteq| 
		& $\supset$ & \Verb|\supset| 
		& $\supseteq$ & \Verb|\supseteq| 
		\\
		\hline
	\end{tabular}
	\caption{Бинарные операторы и операторы сравнения}
\end{table}
	\end{center}
\end{frame}

\begin{frame}{Бинарные операторы, операторы сравнения и стрелки}
	\footnotesize
	\begin{center}
		\begin{table}
	
	\begin{tabular}{|c|c||c|c||c|c||c|c|}
		% first row
		\hline $\to$ & \Verb|\to| 
		& $\longrightarrow$ & \Verb|\longrightarrow| 
		& $\Rightarrow$ & \Verb|\Rightarrow| 
		& $\Longrightarrow$ & \Verb|\Longrightarrow| \\ 
		\hline $\leftarrow$ & \Verb|\leftarrow| 
		& $\longleftarrow$ & \Verb|\longleftarrow| 
		& $\Leftarrow$ & \Verb|\leftarrow| 
		& $\Longleftarrow$ & \Verb|\Longleftarrow|  \\
		\hline
	\end{tabular}
	\caption{Стрелки}
\end{table}
	\end{center}
	
	\normalsize
	
	\LaTeX{} автоматически добавляет пробелы слева и справа от бинарных операторов, операторов сравнения и стрелок.
	
	\begin{minipage}{0.49\textwidth}
		\mintinline{LaTeX}{$x\Leftrightarrow y$}
	\end{minipage}
	\begin{minipage}{0.49\textwidth}
		\Large
		$x\Leftrightarrow y$
	\end{minipage}
	
	\noindent\rule{\textwidth}{1pt}
	\begin{minipage}{0.49\textwidth}
		\mintinline{LaTeX}{$x()y$}
	\end{minipage}
	\begin{minipage}{0.49\textwidth}
		\Large
		$x()y$
	\end{minipage}
\end{frame}

\begin{frame}[fragile=singleslide]{Бинарные операторы, операторы сравнения и стрелки}
	В англоязычной литературе принято использовать $\le$ и $\ge$ (\mintinline{LaTeX}{\le} и \mintinline{LaTeX}{\ge}), а в русскоязычной $\leqslant$ и $\geqslant$ (\mintinline{LaTeX}{\leqslant} и \mintinline{LaTeX}{\geqslant}). Т.к. Дональд Кнут ориентировался в первую очередь на соотечественников, то англоязычные варианты этих знаков имеют более короткие команды.  
	
	Командой \mintinline{LaTeX}{\renewcommand{\le}{\leqslant}} в преамбуле файла можно подменить $\le$ на $\leqslant$ во всем документе.  
\end{frame}


\begin{frame}[fragile=singleslide]{Бинарные операторы, операторы сравнения и стрелки}

	\mintinline{LaTeX}{\mid} тоже считается оператором. Его часто употребляют при определении множеств.
	
	\begin{minipage}{0.49\textwidth}
		\mintinline{LaTeX}{$M=\{x\in A\mid x>0\}$}
	\end{minipage}
	\begin{minipage}{0.49\textwidth}
		$M=\{x\in A\mid x>0\}$
	\end{minipage}
	
	\noindent\rule{\textwidth}{1pt}
	\begin{minipage}{0.49\textwidth}
		\mintinline{LaTeX}{$M=\{x\in A | x>0\}$}
	\end{minipage}
	\begin{minipage}{0.49\textwidth}
		$M=\{x\in A | x>0\}$
	\end{minipage}
	
	\noindent\makebox[\linewidth]{\rule{\paperwidth}{0.4pt}}
	Символ ``\verb|:|'' интерпретируется как оператор деления. При записи отображений необходимо использовать команду ``\verb|\colon|'' вместо  ``\verb|:|''.
	
	\begin{minipage}{0.49\textwidth}
		\mintinline{LaTeX}{$\colon$ X\to Y$}
	
	\end{minipage}
	\begin{minipage}{0.49\textwidth}
		$f\colon X\to Y$
	\end{minipage}
	
	\noindent\rule{\textwidth}{1pt}
	\begin{minipage}{0.49\textwidth}
		\mintinline{LaTeX}{$f: X\to Y$}
	\end{minipage}
	\begin{minipage}{0.49\textwidth}
		$f: X\to Y$
	\end{minipage}
\end{frame}


\begin{frame}{Операторы без пределов}
	\Large
	\begin{center}
		\begin{table}
	\begin{tabular}{||c|c||c|c||c|c||c|c||}
		% first row
		\hline $\log$ & \Verb|\log| 
		& $\lg$ & \Verb|\lg| 
		& $\ln$ & \Verb|\ln| 
		& $\exp$ & \Verb|\exp| \\
		% second row
		\hline  $\sin$ & \Verb|\sin| 
		& $\arcsin$ & \Verb|\arcsin| 
		& $\cos$ & \Verb|\cos| 
		& $\arccos$ & \Verb|\arccos| \\ 
		% third row
		\hline $\tan$ & \Verb|\tan| 
		& $\arctan$ & \Verb|\arctan| 
		& $\cot$ & \Verb|\cot| 
		& & \\
 		% forth row
		\hline\hline \rowcolor{Gray} $\tg$ & \Verb|\tg| 
		& $\arctg$ & \Verb|\arctg| 
		& $\ctg$ & \Verb|\ctg| 
		& $\arcctg$ & \Verb|\arcctg| \\
		% fifth row
		\hline\hline $\sinh$ & \Verb|\sinh| 
		& $\cosh$ & \Verb|\cosh| 
		& $\tanh$ & \Verb|\tanh| 
		& $\coth$ & \Verb|\coth| \\
		% fifth row
		\hline\hline \rowcolor{Gray} $\sh$ & \Verb|\sh| 
		& $\ch$ & \Verb|\ch| 
		& $\th$ & \Verb|\th| 
		& $\cth$ & \Verb|\cth| \\
		% sixth row
		\hline\hline $\arg$ & \Verb|\arg| 
		& $\dim$ & \Verb|\dim|
		& $\sec$ & \Verb|\sec| 
		& $\csc$ & \Verb|\csc| \\ 
		\hline
	\end{tabular}
	\caption{Функции. Команды выделеных строк доступны только при подключении пакета ``babel''}
\end{table}
	\end{center}
\end{frame}

\begin{frame}{Операторы без пределов}
	Все операторы с предыдущего слайда несут свойство \mintinline{TeX}{\nolimits}. Это значит, что если у них поставить индексы, то они появятся в обычном месте, как нижний индекс или как степень.
	
	\begin{minipage}{0.49\textwidth}
		\mintinline{LaTeX}{$\log_{10} x =  \dfrac{\ln x}{\ln 10}$}
	\end{minipage}
	\begin{minipage}{0.49\textwidth}
		\begin{center}
			\Large
			$\log_{10} x =  \dfrac{\ln x}{\ln 10}$
		\end{center}
	\end{minipage}

	\begin{minipage}{0.49\textwidth}
		\mintinline{LaTeX}{$\exp^{-1}x =  \ln x$}
	\end{minipage}
	\begin{minipage}{0.49\textwidth}
		\begin{center}
			\Large
			$\exp^{-1}x =  \ln x$
		\end{center}
	\end{minipage}

	Ещё эти операторы ставят небольшой пробел между функцией и её аргументом.
	
	\begin{minipage}{0.49\textwidth}
		\mintinline{LaTeX}{Сравните $\sin x$ и $\mathrm{sin}x$.}
	\end{minipage}
	\begin{minipage}{0.49\textwidth}
		\begin{center}
			\Large
			Сравните $\sin x$ и $\mathrm{sin}x$.
		\end{center}
	\end{minipage}
	
	Команда \mintinline{TeX}{\operatorname} позволяет создать функцию для одноразового применения.
	
	\begin{minipage}{0.49\textwidth}
		\mintinline{LaTeX}{$\operatorname{arccot}x$.}
	\end{minipage}
	\begin{minipage}{0.49\textwidth}
		\begin{center}
			\Large
			$\operatorname{arccot}x$.
		\end{center}
	\end{minipage}
\end{frame}

\begin{frame}[fragile]{Операторы с пределами}
	\large
	\begin{center}
		\begin{table}
	\begin{tabular}{||c|c||c|c||c|c||c|c||}
		% first row
		\hline $\sum$ & \Verb|\sum| 
		& $\prod$ & \Verb|\prod| 
		& $\bigcup$ & \Verb|\bigcup| 
		& $\bigcap$ & \Verb|\bigcap| \\
		% second row
		\hline  $\min$ & \Verb|\min| 
		& $\inf$ & \Verb|\inf| 
		& $\max$ & \Verb|\max| 
		& $\sup$ & \Verb|\sup| \\ 
		% third row
		\hline $\lim$ & \Verb|\lim| 
		& $\varliminf$ & \Verb|\varliminf| 
		& $\varlimsup$ & \Verb|\varlimsup| 
		& & \\
		\hline
	\end{tabular}
	\caption{Операторы с пределами}
\end{table}
	\end{center}
	
	\normalsize
	\vspace{-0.5cm}
	\begin{minipage}{0.49\textwidth}
		\begin{minted}[autogobble, tabsize=2, frame=single, fontsize=\small]{TeX}
			\[
			\sum_{k=1}^\infty \frac{1}{k^2}
			= \frac{\pi^2}{6}
			\]
			Формулы внутри текста, такие как 
			$\sum_{k=1}^\infty\frac{1}{k^2} 
			= \frac{\pi^2}{6}$,
			занимают меньше места по~высоте.
		\end{minted}
	\end{minipage}
	\begin{minipage}{0.49\textwidth}
		\Large
		\[
		\sum_{k=1}^\infty \frac{1}{k^2}
		= \frac{\pi^2}{6}
		\]
		Формулы внутри текста, такие как 
		$\sum_{k=1}^\infty\frac{1}{k^2} = \frac{\pi^2}{6}$,
		занимают меньше места по~высоте.
	\end{minipage}
	
\end{frame}

\begin{frame}[fragile]{Операторы с пределами}
	Можно все равно заставить выводить индексы в качестве пределов, явно указав пределы после команды \mintinline{TeX}{\limits}.
	
	\begin{minted}[autogobble, tabsize=2, frame=single]{TeX}
		Сумма первых $n$ натуральных чисел выражается формулой 
		$\sum\limits_{i=1}^n i = \dfrac{n(n+1)}{2}$.  
	\end{minted}
	
	
	Сумма первых $n$ натуральных чисел выражается формулой $\sum\limits_{i=1}^n i = \dfrac{n(n+1)}{2}$.
	
\end{frame}



\begin{frame}[fragile]{Операторы с пределами}
	\Large
	\begin{center}
		\begin{table}
	\begin{tabular}{||c|c||c|c||c|c||c|c||}
		% first row
		\hline $\int$ & \Verb|\int| 
		& $\iint$ & \Verb|\iint| 
		& $\iiint$ & \Verb|\iiint| 
		& $\iiiint$ & \Verb|\iiiint| \\
		% second row
		\hline  $\oint$ & \Verb|\oint| 
		& $\idotsint$ & \Verb|\idotsint| 
		& &  
		& & \\ 
		\hline
	\end{tabular}
	\caption{Знаки интегралов}
\end{table}
	\end{center}
	
	\normalsize
	
	У знака интеграла пределы появляются сбоку вне зависимости от режима.
	
	
	\begin{minipage}{0.49\textwidth}
		\begin{minted}[autogobble, tabsize=2, frame=single, fontsize=\small]{TeX}
			\[
			\int_a^b f(x)\,dx
			\] 
		\end{minted}
	\end{minipage}
	\begin{minipage}{0.49\textwidth}
		\Large
		\[
		\int_a^b f(x)\,dx
		\]
	\end{minipage}
	
	\begin{exampleblock}{Замечание}
		``\mintinline{TeX}{\,}'' необходимо в формуле, чтобы поставить правильный пробел между $f(x)$ и $dx$.
	\end{exampleblock}
\end{frame}

\begin{frame}[fragile]{Разные символы}
	\begin{center}
		\begin{table}
	\begin{tabular}{||c|c||c|c||c|c||c|c||}
		% first row
		\hline $\partial$ & \Verb|\partial| 
		& $\prime$ & \Verb|\prime| 
		& $\nabla$ & \Verb|\nabla| 
		& $\infty$ & \Verb|\infty| \\
		% second row
		\hline  $\emptyset$ & \Verb|\emptyset| 
		& $\varnothing$ & \Verb|\varnothing| 
		& $\forall$ & \Verb|\forall| 
		& $\exists$ & \Verb|\exists| \\ 
		% third row
		\hline $\hbar$ & \Verb|\hbar| 
		& $\imath$ & \Verb|\imath| 
		& $\jmath$ & \Verb|\jmath| 
		& $\prime$ & \Verb|\prime| \\ 
		\hline
	\end{tabular}
	\caption{Разные символы}
\end{table}
	\end{center}
	

	\begin{minted}[autogobble, tabsize=2, frame=single, fontsize=\small]{TeX}
		\[
			\dfrac{\partial f(x,y)}{\partial x} =
			\lim_{\Delta x \to 0}\dfrac{f(x + \Delta x,y) - f(x, y)}{\Delta x}
		\] 
	\end{minted}

	\Large
	\[
		\dfrac{\partial f(x, y)}{\partial x} = \lim_{\Delta x\to0} \dfrac{f(x + 	\Delta x, y) - f(x, y)}{\Delta x}
	\] 
\end{frame}


\begin{frame}[fragile]{Скобки}
	\begin{center}
		\begin{table}
	\begin{tabular}{|c|c||c|c||c|c||c|c|}
		% first row
		\hline $($ & \Verb|(| 
		& $)$ & \Verb|)| 
		& $[$ & \Verb|[| 
		& $]$ & \Verb|]| \\
		% second row
		\hline $\{$ & \Verb|{|
		& $\}$ & \Verb|\}| 
		& $|$ & \Verb||| 
		& $\|$ & \Verb|\|| \\
		% third row
		\hline $\langle$ & \Verb|\langle| 
		& $\rangle$ & \Verb|\rangle| 
		& $\backslash$ & \Verb|\backslash|  
		& $/$ & \Verb|/| \\
		% fourth row
		\hline $\lfloor$ & \Verb|\lfloor|
		& $\rfloor$ & \Verb|\rfloor| 
		& $\lceil$ & \Verb|\lceil| 
		& $\rceil$ & \Verb|\rceil| \\ 
		\hline 
	\end{tabular}
	\caption{Скобки}
\end{table}
	\end{center}
	
	\begin{minipage}{0.74\textwidth}
		\begin{minted}[autogobble, tabsize=2, frame=single, fontsize=\small]{TeX}
			\[
			\|\vec{x}-\vec{y}\|_2^2 = \sum_{i=1}^n (x_i - y_i)^2  
			\]
		\end{minted}
	\end{minipage}
	\begin{minipage}{0.25\textwidth}
		\[
			\|\vec{x}-\vec{y}\|_2^2 = \sum_{i=1}^n (x_i - y_i)^2  
		\]
	\end{minipage}
	
	\begin{minipage}{0.74\textwidth}
		\begin{minted}[autogobble, tabsize=2, frame=single, fontsize=\small]{TeX}
			\[
			\|\vec{x}-\vec{y}\|_1^2 = \sum_{i=1}^n |x_i - y_i|  
			\]
		\end{minted}
	\end{minipage}
	\begin{minipage}{0.25\textwidth}
		\[
		\|\vec{x}-\vec{y}\|_1^2 = \sum_{i=1}^n |x_i - y_i|  
		\]
	\end{minipage}
	
\end{frame}



\begin{frame}[fragile]{Скобки}
	Команды \mintinline{TeX}{\left} и \mintinline{TeX}{\right} позволяют автоматически подбирать размер скобки под габариты содержимого между ними. 
	
	\begin{minipage}{0.72\textwidth}
		\begin{minted}[autogobble, tabsize=2, frame=single, fontsize=\small]{TeX}
			\[
			e = \lim_{n\to\infty}(1 + \dfrac{1}{n})^n
			\]
		\end{minted}
	\end{minipage}
	\begin{minipage}{0.27\textwidth}
		\Large
		\[
		e = \lim_{n\to\infty}(1 + \dfrac{1}{n})^n
		\]
	\end{minipage}
	
	\begin{minipage}{0.72\textwidth}
		\begin{minted}[autogobble, tabsize=2, frame=single, fontsize=\small]{TeX}
			\[
			e = \lim_{n\to\infty}\left(1 + \dfrac{1}{n}\right)^n
			\]
		\end{minted}
	\end{minipage}
	\begin{minipage}{0.27\textwidth}
		\Large
		\[
		e = \lim_{n\to\infty}\left(1 + \dfrac{1}{n}\right)^n
		\]
	\end{minipage}

	Эти команды всегда должны идти парой!
	
\end{frame}


\begin{frame}[fragile]{Скобки}
	Знак деления является скобкой неслучайно: иногда хочется увеличивать его.
	
	\begin{minipage}{0.54\textwidth}
		\begin{minted}[autogobble, tabsize=2, frame=single, fontsize=\small]{TeX}
			\[M(f)=
			\left(\int\limits_a^bf(x)\,dx\right)
			/(b-a)\]
		\end{minted}
	\end{minipage}
	\begin{minipage}{0.44\textwidth}
		\[
		M(f)=\
		\left(\int\limits_a^bf(x)\,dx\right)
		/(b-a)
		\]
	\end{minipage}

	Используя точку вместо левой скобки, можно указать \LaTeX{}, под размер чего подгонять правую скобку.
	
	\begin{minipage}{0.54\textwidth}
		\begin{minted}[autogobble, tabsize=2, frame=single, fontsize=\small]{TeX}
			\[
				M(f)=\left.
				\left(\int\limits_a^bf(x)\,dx\right)
				\right/(b-a)
			\]
		\end{minted}
	\end{minipage}
	\begin{minipage}{0.44\textwidth}
		\[
		M(f)=\left.
		\left(\int\limits_a^bf(x)\,dx\right)
		\right/(b-a)
		\]
	\end{minipage}
	
\end{frame}

\begin{frame}[fragile]{Скобки}
	
	\large
	\begin{minted}[autogobble, tabsize=2, frame=single]{TeX}
		\[
		\int\limits_a^b x^{-2}\,dx = -\left.\dfrac{1}{x}\right|_a^b
		\]
	\end{minted}
	\Large
	\[
		\int\limits_a^b x^{-2}\,dx = -\left.\dfrac{1}{x}\right|_a^b
	\]
	

	
\end{frame}



\begin{frame}[fragile]{Скобки}
	Иногда автоматически подобранный размер неудачен. В таких случаях можно в ручную указать конкретный размер скобки.
	
	\begin{center}
		\begin{table}
	\begin{tabular}{|c|c||c|c||c|c||c|c|}
		% first row
		\hline $\bigl($ & \Verb|\bigl(|
		& $\Bigl($ & \Verb|\Bigl(| 
		& $\biggl($ & \Verb|\biggl(|
		& $\Biggl($ & \Verb|\Biggl(| \\ 
		
		% second row
		\hline $\bigr)$ & \Verb|\bigr)| 
		& $\Bigr)$ & \Verb|\Bigr)| 
		& $\biggr)$ & \Verb|\biggr)| 
		& $\Biggr)$ & \Verb|\Biggr)| \\ 
		\hline 
	\end{tabular}
	\caption{Увеличенные кобки}
\end{table}
	\end{center}
	
	\large
	
	\begin{minipage}{0.49\textwidth}
		\begin{minted}[autogobble, tabsize=2, frame=single]{TeX}
			$$\left||x+1| - |x-1|\right|$$
		\end{minted}
	\end{minipage}
	\begin{minipage}{0.49\textwidth}
		$$\left||x+1| - |x-1|\right|$$
	\end{minipage}

	\begin{minipage}{0.49\textwidth}
		\begin{minted}[autogobble, tabsize=2, frame=single]{TeX}
			$$\bigl||x+1| - |x-1|\bigr|$$
		\end{minted}
	\end{minipage}
	\begin{minipage}{0.49\textwidth}
		$$\bigl||x+1| - |x-1|\bigr|$$
	\end{minipage}
	
\end{frame}


\begin{frame}[fragile]{Диакритические знаки}
	\large
	\begin{center}
		\begin{table}
	\begin{tabular}{|c|c||c|c||c|c||c|c||c|c|}
		% first row
		\hline $\hat a$ & \Verb|\hat a| 
		& $\check a$ & \Verb|\check a| 
		& $\tilde a$ & \Verb|\tilde a| 
		& $\acute a$ & \Verb|\acute a| 
		& $\grave a$ & \Verb|\grave a| \\
		% second row
		\hline $\dot a$ & \Verb|\dot a| 
		& $\ddot a$ & \Verb|\ddot a| 
		& $\breve a$ & \Verb|\breve a| 
		& $\bar a$ & \Verb|\bar a| 
		& $\vec a$ & \Verb|\vec a| \\
		\hline
	\end{tabular}
	\caption{Дакритические знаки}
\end{table}
	\end{center}
	
	
	\begin{minipage}{0.59\textwidth}
		\begin{minted}[autogobble, tabsize=2, frame=single]{TeX}
			\[
			\ddot a \eqdef \dfrac{d^2 a}{dt^2}
			\]
		\end{minted}
	\end{minipage}
	\begin{minipage}{0.39\textwidth}
		\Large
		\[
		\ddot a \eqdef \dfrac{d^2 a}{dt^2}
		\]
	\end{minipage}

	\begin{minipage}{0.59\textwidth}
		\begin{minted}[autogobble, tabsize=2, frame=single]{TeX}
			\[
			\bar x =  \dfrac1n \sum_{i=1}^n x_i
			\]
		\end{minted}
	\end{minipage}
	\begin{minipage}{0.39\textwidth}
		\Large
		\[
		\bar x =  \dfrac1n \sum_{i=1}^n x_i
		\]
	\end{minipage}
\end{frame}

\begin{frame}[fragile]{Диакритические знаки}
	\large
	
	Диакритические знаки с прошлого слайда имеют фиксированную ширину.
	
	\begin{minipage}{0.59\textwidth}
		\mintinline{TeX}{$\hat{f*g}=\hat f\cdot\hat g$}
	\end{minipage}
	\begin{minipage}{0.39\textwidth}
		\begin{center}
			$\hat{f*g}=\hat f\cdot\hat g$
		\end{center}
	\end{minipage}
	
	\begin{minipage}{0.59\textwidth}
		\mintinline{TeX}{$\vec{AC} = \vec{AB} + \vec{BC}$}
	\end{minipage}
	\begin{minipage}{0.39\textwidth}
		\begin{center}
			$\vec{AC} = \vec{AB} + \vec{BC}$
		\end{center}
	\end{minipage}

	В таких ситуациях лучше использовать команды \mintinline{TeX}{\widehat}, \mintinline{TeX}{\overline}, \mintinline{TeX}{\overrightarrow}  и т.п.

	\begin{minipage}{0.59\textwidth}
		\mintinline{TeX}{$\widehat{f*g}=\hat f\cdot\hat g$}
	\end{minipage}
	\begin{minipage}{0.39\textwidth}
		\begin{center}
			$\widehat{f*g}=\hat f\cdot\hat g$
		\end{center}
	\end{minipage}

	\begin{minipage}{0.59\textwidth}
		\begin{minted}[autogobble, tabsize=2, frame=single]{TeX}
			\[
			\overrightarrow{AC} = 
			\overrightarrow{AB} + 
			\overrightarrow{BC}
			\]
		\end{minted}
	\end{minipage}
	\begin{minipage}{0.39\textwidth}
		\begin{center}
			\[
			\overrightarrow{AC} = 
			\overrightarrow{AB} + 
			\overrightarrow{BC}
			\]
		\end{center}
	\end{minipage}	
\end{frame}


\begin{frame}[fragile]{Матрицы}
	Матрицы с круглыми скобками набираются с помощью окружения \mintinline{TeX}{pmatrix}. Матрицы задаются по строкам, строки отделятся с помощью \mintinline{TeX}{\\}, а столбцы с помощью \mintinline{TeX}{&}. 
	
	\begin{minipage}{0.59\textwidth}
		\begin{minted}[autogobble, tabsize=2, frame=single,  fontsize=\small]{TeX}
			\[
				\begin{pmatrix}
					a_1 & a_2 
				\end{pmatrix}
			\]
		\end{minted}
	\end{minipage}
	\begin{minipage}{0.39\textwidth}
		\begin{center}
			\LARGE
			\[
			\begin{pmatrix}
				a_1 & a_2 
			\end{pmatrix}
			\]
		\end{center}
	\end{minipage}	

	\begin{minipage}{0.59\textwidth}
		\begin{minted}[autogobble, tabsize=2, frame=single, fontsize=\small]{TeX}
			\[
			\begin{pmatrix}
				a_1 \\ a_2 
			\end{pmatrix}
			\]
		\end{minted}
	\end{minipage}
	\begin{minipage}{0.39\textwidth}
		\begin{center}
			\LARGE
			\[
			\begin{pmatrix}
				a_1 \\ a_2 
			\end{pmatrix}
			\]
		\end{center}
	\end{minipage}
\end{frame}

\begin{frame}[fragile]{Матрицы: \mintinline{TeX}{pmatrix}}
	\begin{minted}[autogobble, tabsize=2, frame=single]{TeX}
	\[
	\begin{pmatrix}
			a_{11} - \lambda & a_{12} & a_{13} \\
			a_{21} & a_{22} - \lambda & a_{23} \\
			a_{31} & a_{32} & a_{33} - \lambda
	\end{pmatrix}
	\]
	\end{minted}
	\Large
	\[
	\begin{pmatrix}
		a_{11} - \lambda & a_{12} & a_{13} \\
		a_{21} & a_{22} - \lambda & a_{23} \\
		a_{31} & a_{32} & a_{33} - \lambda
	\end{pmatrix}
	\]
\end{frame}


\begin{frame}[fragile]{Матрицы: \mintinline{TeX}{matrix}}
	\begin{minted}[autogobble, tabsize=2, frame=single]{TeX}
		\[
		\begin{matrix}
			a_{11} - \lambda & a_{12} & a_{13} \\
			a_{21} & a_{22} - \lambda & a_{23} \\
			a_{31} & a_{32} & a_{33} - \lambda
		\end{matrix}
		\]
	\end{minted}
	\Large
	\[
	\begin{matrix}
		a_{11} - \lambda & a_{12} & a_{13} \\
		a_{21} & a_{22} - \lambda & a_{23} \\
		a_{31} & a_{32} & a_{33} - \lambda
	\end{matrix}
	\]
\end{frame}

\begin{frame}[fragile]{Матрицы: \mintinline{TeX}{vmatrix}}
	\begin{minted}[autogobble, tabsize=2, frame=single]{TeX}
		\[
		\begin{vmatrix}
			a_{11} - \lambda & a_{12} & a_{13} \\
			a_{21} & a_{22} - \lambda & a_{23} \\
			a_{31} & a_{32} & a_{33} - \lambda
		\end{vmatrix}
		\]
	\end{minted}
	\Large
	\[
	\begin{vmatrix}
		a_{11} - \lambda & a_{12} & a_{13} \\
		a_{21} & a_{22} - \lambda & a_{23} \\
		a_{31} & a_{32} & a_{33} - \lambda
	\end{vmatrix}
	\]
\end{frame}

\begin{frame}[fragile]{Матрицы: \mintinline{TeX}{Vmatrix}}
	\begin{minted}[autogobble, tabsize=2, frame=single]{TeX}
		\[
		\begin{Vmatrix}
			a_{11} - \lambda & a_{12} & a_{13} \\
			a_{21} & a_{22} - \lambda & a_{23} \\
			a_{31} & a_{32} & a_{33} - \lambda
		\end{Vmatrix}
		\]
	\end{minted}
	\Large
	\[
	\begin{Vmatrix}
		a_{11} - \lambda & a_{12} & a_{13} \\
		a_{21} & a_{22} - \lambda & a_{23} \\
		a_{31} & a_{32} & a_{33} - \lambda
	\end{Vmatrix}
	\]
\end{frame}

\begin{frame}[fragile]{Матрицы: \mintinline{TeX}{bmatrix}}
	\begin{minted}[autogobble, tabsize=2, frame=single]{TeX}
		\[
		\begin{bmatrix}
			a_{11} - \lambda & a_{12} & a_{13} \\
			a_{21} & a_{22} - \lambda & a_{23} \\
			a_{31} & a_{32} & a_{33} - \lambda
		\end{bmatrix}
		\]
	\end{minted}
	\Large
	\[
	\begin{bmatrix}
		a_{11} - \lambda & a_{12} & a_{13} \\
		a_{21} & a_{22} - \lambda & a_{23} \\
		a_{31} & a_{32} & a_{33} - \lambda
	\end{bmatrix}
	\]
\end{frame}


\begin{frame}[fragile]{Матрицы: \mintinline{TeX}{bmatrix}}
	Матрицы входят в формулы как обычные символы крупного размера.
	\begin{minted}[autogobble, tabsize=2, frame=single]{TeX}
		\[
		\vec a = \begin{bmatrix}
			a_x \\
			a_y \\
			a_z 
		\end{bmatrix}
		\]
	\end{minted}
	\Large
	\[
	\vec a = 
	\begin{bmatrix}
		a_x \\
		a_y \\
		a_z 
	\end{bmatrix}
	\]
\end{frame}


\begin{frame}[fragile]{Матрицы: \mintinline{TeX}{bmatrix}}
	Многоточия в матрице можно поставить командами \mintinline{TeX}{\cdots}, \mintinline{TeX}{\vdots} и \mintinline{TeX}{\ddots}.
	
	\begin{minted}[autogobble, tabsize=2, frame=single]{TeX}
		\[
		\begin{vmatrix}
			1 & \cdots & 1 \\
			\vdots & \ddots & \vdots \\
			1 & \cdots & 1  
		\end{vmatrix} = 0
		\]
	\end{minted}
	\Large
	\[
	\begin{vmatrix}
		1 & \cdots & 1 \\
		\vdots & \ddots & \vdots \\
		1 & \cdots & 1  
	\end{vmatrix} = 0
	\]
\end{frame}