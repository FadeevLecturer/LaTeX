\section{Пользовательские команды}

\begin{frame}[fragile]{Определение пользовательских функций}

	В \LaTeX{} нет встроенных команд для принятых в России сокращений ряда функций. Например, действительная часть комплексного числа обозначается зарубежом $\Re(z)$ (\mintinline{LaTeX}{$\Re(z)$}), а комплексная часть $\Im(z)$ (\mintinline{LaTeX}{$\Im(z)$}). В России же приняты обозначения $\re z$ и $\im z$, но соответствующих команд в \LaTeX{} нет.
	
	
	\begin{minipage}{0.49\textwidth}
		\begin{minted}[autogobble, frame=single]{TeX}
			$Re z$
		\end{minted}
	\end{minipage}
	\begin{minipage}{0.49\textwidth}
		\begin{center}
			$Re z$
		\end{center}
	\end{minipage}
	
	Не должно быть наклона. \mintinline{LaTeX}{\mathsf} позволяет писать прямым шрифтом.
	
	\begin{minipage}{0.49\textwidth}
		\begin{minted}[autogobble, frame=single]{TeX}
			$\mathsf{Re} z$
		\end{minted}
	\end{minipage}
	\begin{minipage}{0.49\textwidth}
		\begin{center}
			$\mathsf{Re} z$
		\end{center}
	\end{minipage}

	Между функцией и её аргументом должен быть небольшой пробел.  \mintinline{LaTeX}{\mathop} решает эту проблему.

	\begin{minipage}{0.49\textwidth}
		\begin{minted}[autogobble, frame=single]{TeX}
			$\mathop{\mathsf{Re}} z$
		\end{minted}
	\end{minipage}
	\begin{minipage}{0.49\textwidth}
		\begin{center}
			$\mathop{\mathsf{Re}} z$
		\end{center}
	\end{minipage}
	
\end{frame}

\begin{frame}[fragile]{Определение пользовательских функций}
	Если в тексте часто встречается взятие действительной части комплексного числа, то писать каждый раз \mintinline{LaTeX}{\mathop{\mathsf{Re}}} становится утомительно и чревато ошибками.  Это самая простая ситуация, когда полезно определить команду для пользовательской функции командой \mintinline{LaTeX}{\DeclareMathOperator{cmd}{text}} \textbf{в преамбуле документа}.
	\begin{minted}[autogobble, frame=single]{TeX}
		% preambule
		\DeclareMathOperator{\re}{Re}
		\DeclareMathOperator{\im}{Im}
	\end{minted}
	
	Указав две строки выше \textbf{в преамбуле}, можно в тексте использовать команды \mintinline{LaTeX}{\re} и \mintinline{LaTeX}{\im}.
	
	\begin{minipage}{0.49\textwidth}
		\begin{minted}[autogobble, frame=single]{TeX}
			$z = \re z + i \im z$ 
		\end{minted}
	\end{minipage}
	\begin{minipage}{0.49\textwidth}
		\begin{center}
			$z = \re z + i \im z$
		\end{center}	 
	\end{minipage}		
\end{frame}


\begin{frame}[fragile]{Определение пользовательских команд}
	
	Комманда \mintinline{LaTeX}{\newcommand} позволяет определять пользовательские команды-макросы. Например, рассмотрим символ $\stackrel{\mathrm{def}}{=}$, который получается с помощью выражения \mintinline{LaTeX}{\stackrel{\mathrm{def}}{=}}.
	
	\begin{minipage}{0.69\textwidth}
		\begin{minted}[autogobble, frame=single]{TeX}
			$x^2\stackrel{\mathrm{def}}{=} x \cdot x$
		\end{minted}
	\end{minipage}
	\begin{minipage}{0.29\textwidth}
		\begin{center}
			$x^2\eqdef x \cdot x$
		\end{center}
	\end{minipage}
	
	 Следующей командой в преамбуле документа определяется новая команда \mintinline{LaTeX}{\eqdef}, которая эквивалента сочетанию \mintinline{LaTeX}{\stackrel{\mathrm{def}}{=}}:

	\begin{minted}[autogobble, frame=single]{TeX}
		% preambule
		\newcommand{\eqdef}{\stackrel{\mathrm{def}}{=}}
	\end{minted}

	Это сокращает запись предыдущей формулы.
	
	\begin{minipage}{0.69\textwidth}
		\begin{minted}[autogobble, frame=single]{TeX}
			$x^2\eqdef x \cdot x$
		\end{minted}
	\end{minipage}
	\begin{minipage}{0.29\textwidth}
		\begin{center}
			$x^2\eqdef x \cdot x$
		\end{center}
	\end{minipage}


\end{frame}

\begin{frame}[fragile]{Определение пользовательских команд}
	Так, например, команда  \mintinline{LaTeX}{\DeclareMathOperator} на самом деле под капотом работает превращается в \mintinline{LaTeX}{\newcommand} с небольшими изменениями в аргумент. Например, следующее определение для действительной части комплексного числа 
	
	\begin{minted}[autogobble, frame=single]{TeX}
		\DeclareMathOperator{\re}{Re}
	\end{minted}
	\vspace{-0.3cm}
	в точности соответствует 
	\begin{minted}[autogobble, frame=single]{TeX}
		\newcommand{\re}{\mathop{\mathrm{Re}}\nolimits}
	\end{minted}
	
	Здесь \mintinline{LaTeX}{\nolimits} влияет на то, как будут отображаться индексы, что в данном случае не принципиально. 
	
	\begin{center}
		\begin{tabular}{|c|c|}
			\hline 
			\mintinline{LaTeX}{\nolimits} & \mintinline{LaTeX}{\limits} \\
			\hline
			$\re_i^j$ & $\mathop{\mathsf{Re}}\limits_i^j$ \\
			\hline
		\end{tabular}
	\end{center}
	
\end{frame}


\begin{frame}[fragile]{Определение пользовательских команд}
	Предположим, что в тексте часто встречается действительная ось $\mathbb{R}$, которая получается командой \mintinline{LaTeX}{\mathbb{R}} и мы хотим определить команду \mintinline{LaTeX}{\R} для этого. 
	
	Такая команда будет работать только внутри формулы.
	\begin{minted}[autogobble, frame=single]{TeX}
		\newcommand{\R}{\mathbb{R}}
	\end{minted}

	Такая~--- только снаружи.
	\begin{minted}[autogobble, frame=single]{TeX}
		\newcommand{\R}{$\mathbb{R}$}
	\end{minted}
	
	А такая~--- везде.
	\begin{minted}[autogobble, frame=single]{TeX}
		\newcommand{\R}{\ensuremath{\mathbb{R}}}
	\end{minted}
\end{frame}

\begin{frame}[fragile]{Пользовательские команды с аргументами}
	Можно определять и команды с аргументами. Тогда необходимо указать количество аргументов в квадратных скобках, а ссылаться на них внутри команды можно с помощью символа ``\#'' и порядкового номера после неё.
	
	Например, следующее определение 
	\begin{minted}[autogobble, frame=single]{TeX}
		% preambule
		\newcommand{\ton}[1]{1, 2, \dots, #1}
	\end{minted}
	сокращает запись формул вида
	\begin{minted}[autogobble, frame=single]{TeX}
		Здесь $a_ij, i=\ton{n},\, j=\ton{m}$~--- элементы матрицы $A$.
	\end{minted}
	Здесь $a_{ij},\, i=\ton{n},\, j=\ton{m}$~--- элементы матрицы $A$.	
\end{frame}

\begin{frame}[fragile]{Пользовательские команды с аргументами}

	Запись уравнений в частных производных значительно упрощается, если определить команду \mintinline{LaTeX}{\dd} следующим образом.
	\begin{minted}[autogobble, frame=single]{TeX}
		% preambule
		\newcommand{\dd}[2]{\dfrac{\partial #1}{\partial #2}}
	\end{minted}
	
	Тогда уравнение $\dd{\psi}{t} + u\dd{\psi}{x}=0$ набирается с помощью 
	
	\begin{minted}[autogobble, frame=single]{TeX}
			$\dd{\psi}{t} + u\dd{\psi}{x}=0$
	\end{minted}
	
	вместо 
	
	\begin{minted}[autogobble, frame=single, fontsize=\small]{TeX}
		$\dfrac{\partial\psi}{\partial t} + u\dfrac{\partial\psi}{\partial x} = 0$
	\end{minted}
\end{frame}

\begin{frame}[fragile]{Переопределение команды}
	Если команда с таким именем уже существует, то попытка переопределить её через \mintinline{LaTeX}{\newcommand} приведет к ошибке. Если все же есть такая необходимость, то можно воспользоваться альтернативной командой \mintinline{LaTeX}{\renewcommand}.
	
	Например, в русскоязычных текстах часто переопределяют команды \mintinline{LaTeX}{\le} и \mintinline{LaTeX}{\ge} на \mintinline{LaTeX}{\leqslant} и \mintinline{LaTeX}{\geslant} соответственно.
	
	\begin{minipage}{0.49\textwidth}
		\begin{minted}[autogobble, frame=single]{TeX}
			Пусть $x\ge0$ и $y\le0$.
		\end{minted}
	\end{minipage}
	\begin{minipage}{0.49\textwidth}
		\begin{center}
			Пусть $x\ge0$ и $y\le0$.
		\end{center}
	\end{minipage}

	\begin{minted}[autogobble, frame=single]{TeX}
		% preambule
		\renewcommand{\le}{\leqslant}
		\renewcommand{\ge}{\geqslant}
	\end{minted}

	\begin{minipage}{0.49\textwidth}
		\begin{minted}[autogobble, frame=single]{TeX}
			Пусть $x\ge0$ и $y\le0$.
		\end{minted}
	\end{minipage}
	\begin{minipage}{0.49\textwidth}
		\begin{center}
			Пусть $x\geqslant0$ и $y\leqslant0$.
		\end{center}
	\end{minipage}
	
\end{frame}